% Options for packages loaded elsewhere
\PassOptionsToPackage{unicode}{hyperref}
\PassOptionsToPackage{hyphens}{url}
%
\documentclass[
]{book}
\usepackage{amsmath,amssymb}
\usepackage{iftex}
\ifPDFTeX
  \usepackage[T1]{fontenc}
  \usepackage[utf8]{inputenc}
  \usepackage{textcomp} % provide euro and other symbols
\else % if luatex or xetex
  \usepackage{unicode-math} % this also loads fontspec
  \defaultfontfeatures{Scale=MatchLowercase}
  \defaultfontfeatures[\rmfamily]{Ligatures=TeX,Scale=1}
\fi
\usepackage{lmodern}
\ifPDFTeX\else
  % xetex/luatex font selection
\fi
% Use upquote if available, for straight quotes in verbatim environments
\IfFileExists{upquote.sty}{\usepackage{upquote}}{}
\IfFileExists{microtype.sty}{% use microtype if available
  \usepackage[]{microtype}
  \UseMicrotypeSet[protrusion]{basicmath} % disable protrusion for tt fonts
}{}
\makeatletter
\@ifundefined{KOMAClassName}{% if non-KOMA class
  \IfFileExists{parskip.sty}{%
    \usepackage{parskip}
  }{% else
    \setlength{\parindent}{0pt}
    \setlength{\parskip}{6pt plus 2pt minus 1pt}}
}{% if KOMA class
  \KOMAoptions{parskip=half}}
\makeatother
\usepackage{xcolor}
\usepackage{longtable,booktabs,array}
\usepackage{calc} % for calculating minipage widths
% Correct order of tables after \paragraph or \subparagraph
\usepackage{etoolbox}
\makeatletter
\patchcmd\longtable{\par}{\if@noskipsec\mbox{}\fi\par}{}{}
\makeatother
% Allow footnotes in longtable head/foot
\IfFileExists{footnotehyper.sty}{\usepackage{footnotehyper}}{\usepackage{footnote}}
\makesavenoteenv{longtable}
\usepackage{graphicx}
\makeatletter
\def\maxwidth{\ifdim\Gin@nat@width>\linewidth\linewidth\else\Gin@nat@width\fi}
\def\maxheight{\ifdim\Gin@nat@height>\textheight\textheight\else\Gin@nat@height\fi}
\makeatother
% Scale images if necessary, so that they will not overflow the page
% margins by default, and it is still possible to overwrite the defaults
% using explicit options in \includegraphics[width, height, ...]{}
\setkeys{Gin}{width=\maxwidth,height=\maxheight,keepaspectratio}
% Set default figure placement to htbp
\makeatletter
\def\fps@figure{htbp}
\makeatother
\setlength{\emergencystretch}{3em} % prevent overfull lines
\providecommand{\tightlist}{%
  \setlength{\itemsep}{0pt}\setlength{\parskip}{0pt}}
\setcounter{secnumdepth}{5}
\usepackage{booktabs}
\ifLuaTeX
  \usepackage{selnolig}  % disable illegal ligatures
\fi
\usepackage[]{natbib}
\bibliographystyle{plainnat}
\IfFileExists{bookmark.sty}{\usepackage{bookmark}}{\usepackage{hyperref}}
\IfFileExists{xurl.sty}{\usepackage{xurl}}{} % add URL line breaks if available
\urlstyle{same}
\hypersetup{
  hidelinks,
  pdfcreator={LaTeX via pandoc}}

\author{}
\date{\vspace{-2.5em}2024-01-26}

\begin{document}

{
\setcounter{tocdepth}{1}
\tableofcontents
}
\hypertarget{about}{%
\chapter*{About}\label{about}}
\addcontentsline{toc}{chapter}{About}

Welcome to SOST70023 Data Cleaning and Visualisation using R! This notebook will host the materials for all practical exercises for this course unit.

\hypertarget{how-to-use-this-notebook}{%
\chapter*{How to use this notebook}\label{how-to-use-this-notebook}}
\addcontentsline{toc}{chapter}{How to use this notebook}

\hypertarget{introducing-statistical-programming-with-r-and-rstudio}{%
\chapter{Introducing Statistical Programming with R and RStudio}\label{introducing-statistical-programming-with-r-and-rstudio}}

To complete the exercises, ensure that you work in your newly created project in RStudio and writing your answers in your new R script.

\hypertarget{creating-and-exploring-vector-objects}{%
\section{Creating and Exploring Vector Objects}\label{creating-and-exploring-vector-objects}}

\hypertarget{exercise-1.1}{%
\subsection{Exercise 1.1}\label{exercise-1.1}}

Create a vector object that contains the following values: 70, 8, 50, 100. Name this vector \texttt{num\_vct} and print the contents.

\begin{enumerate}
\def\labelenumi{\alph{enumi}.}
\tightlist
\item
  Use the class() function to find out what type of object \texttt{num\_vct} is.
\item
  Multiply this vector by 2.
\item
  Divide the vector by 10.
\item
  Subtract 5 from the vector.
\item
  Add 8 to the vector.
\end{enumerate}

\hypertarget{exercise-1.2}{%
\subsection{Exercise 1.2}\label{exercise-1.2}}

Create a character vector object that contains the following: I, love, programming, with, R. Name this vector \texttt{char\_vct} and print the contents.

\begin{enumerate}
\def\labelenumi{\alph{enumi}.}
\tightlist
\item
  Use the class() function to find out what type of object \texttt{char\_vct} is.
\item
  Perform tasks b to e from Exercise 1. What results do you obtain? Why?
\end{enumerate}

\hypertarget{exercise-1.3}{%
\subsection{Exercise 1.3}\label{exercise-1.3}}

Create a new numeric vector that contains the following values: 5, 7, 9, 10. Name it \texttt{num\_vct2} and print the contents.

\begin{enumerate}
\def\labelenumi{\alph{enumi}.}
\tightlist
\item
  Add num\_vct2 to num\_vct.
\item
  Divide num\_vct by num\_vct2.
\item
  Multiply num\_vct2 by num\_vct.
\item
  Subtract num\_vct from num\_vct2
\end{enumerate}

\hypertarget{exercise-1.4}{%
\subsection{Exercise 1.4}\label{exercise-1.4}}

The number of elements in a vector is referred to as the length of the vector. With small vectors such as those you have already created, you can simply count these. Alternatively, the \texttt{length()} function can be used.

Use the length function to obtain the length of the following objects: \texttt{num\_vct}, \texttt{char\_vct}, \texttt{num\_vct2}.

\hypertarget{exercise-1.5}{%
\subsection{Exercise 1.5}\label{exercise-1.5}}

Create a vector object that contains the following logical and numeric values: TRUE, 6, FALSE, 10, FALSE. Name it \texttt{new\_vct} and print the output.

Does the output match the contents you entered when you created the vector? Why?

\hypertarget{importing-and-exporting-data}{%
\section{Importing and Exporting Data}\label{importing-and-exporting-data}}

Throughout the course units of this programme, you will utilise real datasets to develop your data analysis and interpretation skills.

For this exercise, you will navigate to the UK Data Service (UKDS) website and acquire the \textit{British Social Attitudes Survey (2019), Poverty and Welfare: Open Access Teaching Dataset}.

\hypertarget{exercise-2.1}{%
\subsection{Exercise 2.1}\label{exercise-2.1}}

Navigate to the British Social Attitudes Survey (2019) webpage on the UK Data Service \href{https://beta.ukdataservice.ac.uk/datacatalogue/studies/study?id=8850#!/access-data}{website}.

Scroll down to the end of the webpage and download the dataset in both SPSS and STATA formats in your R project working directory.

To access the SPSS data file, unzip folder and open the folders in the following sequence: UKDA-8850-spss \textgreater{} spss \textgreater{} spss25. Follow the same approach for accessing the STATA file. Place these two files in your root R project working directory or within a sub-folder in this directory.

\begin{enumerate}
\def\labelenumi{\alph{enumi}.}
\tightlist
\item
  Import the dataset in SPSS format; name this object \texttt{bsas\_spss}.
\item
  Import the dataset in STATA format; name this object \texttt{bsas\_stata}.
\end{enumerate}

\hypertarget{exercise-2.2}{%
\subsection{Exercise 2.2}\label{exercise-2.2}}

\begin{enumerate}
\def\labelenumi{\alph{enumi}.}
\tightlist
\item
  How many observations and variables do the \texttt{bsas\_spss} and \texttt{bsas\_stata} objects have? Are these identical for both objects?
\item
  Use the view function to explore these data objects.
\end{enumerate}

\hypertarget{exercise-2.3}{%
\subsection{Exercise 2.3}\label{exercise-2.3}}

Export \textbf{both} \texttt{bsas\_spss} and \texttt{bsas\_stata} data objects in \texttt{.RData} format.

\hypertarget{exercise-2.4}{%
\subsection{Exercise 2.4}\label{exercise-2.4}}

Export \texttt{bsas\_spss} as a \textbf{single} object in \texttt{.RDS} format.

\hypertarget{section}{%
\section{\texorpdfstring{\underline{Exercise 2.5}}{}}\label{section}}

Export the \texttt{bsas\_stata} object as a \texttt{.csv} file.

\hypertarget{section-1}{%
\section{\texorpdfstring{\underline{Exercise 2.6}}{}}\label{section-1}}

\begin{enumerate}
\def\labelenumi{\alph{enumi}.}
\tightlist
\item
  Import the \texttt{.csv} file you have created in 2.5 and name this object \texttt{bsas\_stata2}. View the contents of the \texttt{bsas\_stata2} and \texttt{bsas\_stata} objects. What are the differences between the two, if any? Why?
\item
  Import the file created in 2.4 and name this object \texttt{bsas\_spss2}.
\item
  Load the file created in 2.3.
\end{enumerate}

\hypertarget{bonus-task-importing-data-from-github-repositories}{%
\section{Bonus Task: Importing Data from Github Repositories}\label{bonus-task-importing-data-from-github-repositories}}

Many different types of data can be imported in R using either base R functions or functions from packages. However, R is not limited to importing `hard copy' files from your machine but also supports direct import of data files located on websites for example.

A Github Repository is a cloud-based, online platform that allows programming users to openly share research projects, associated documentation, data files, and comments with other users and/or the public.
Explore the \href{(https://github.com/CSSEGISandData/COVID-19)}{Johns Hopkins Whiting School of Engineering COVID-19 Github Repository}.

\begin{enumerate}
\def\labelenumi{\arabic{enumi}.}
\tightlist
\item
  Open the csse\_covid\_19\_data folder found at the beginning of the page.
\item
  Then open the csse\_covid\_19\_time\_series folder.
\item
  Click on time\_series\_covid19\_confirmed\_global.csv link.
\item
  Click on View Raw.
\item
  The webpage will now show you a long series of numbers. This is the `data file' that you need to import. Note that this file is a \texttt{.csv} file. The same base R function you have previously used will also work in this case.
\end{enumerate}

To import the data file, create a new object called \texttt{covid\_johnshop}. Within the base R \texttt{.csv} function, paste the full data file web link. Do not forget to enclose this link with quotation marks.

View the contents of this object. How many observations and variables does this object have? What class is this object?

\hypertarget{the-structure-of-data-objects}{%
\chapter{The Structure of Data Objects}\label{the-structure-of-data-objects}}

\hypertarget{manipulating-and-tidying-data}{%
\chapter{Manipulating and Tidying Data}\label{manipulating-and-tidying-data}}

You can add parts to organize one or more book chapters together. Parts can be inserted at the top of an .Rmd file, before the first-level chapter heading in that same file.

Add a numbered part: \texttt{\#\ (PART)\ Act\ one\ \{-\}} (followed by \texttt{\#\ A\ chapter})

Add an unnumbered part: \texttt{\#\ (PART\textbackslash{}*)\ Act\ one\ \{-\}} (followed by \texttt{\#\ A\ chapter})

Add an appendix as a special kind of un-numbered part: \texttt{\#\ (APPENDIX)\ Other\ stuff\ \{-\}} (followed by \texttt{\#\ A\ chapter}). Chapters in an appendix are prepended with letters instead of numbers.

\hypertarget{reproducible-workflows-with-r-markdown}{%
\chapter{Reproducible Workflows with R Markdown}\label{reproducible-workflows-with-r-markdown}}

\hypertarget{data-visualisation}{%
\chapter{Data Visualisation}\label{data-visualisation}}

\hypertarget{working-with-special-data-types-text-data}{%
\chapter{Working with Special Data Types: Text Data}\label{working-with-special-data-types-text-data}}

\hypertarget{programming-with-r}{%
\chapter{Programming with R}\label{programming-with-r}}

\hypertarget{practising-data-cleaning-and-visualisation-formative-peer-review-case-study}{%
\chapter{Practising Data Cleaning and Visualisation: Formative Peer-Review Case Study}\label{practising-data-cleaning-and-visualisation-formative-peer-review-case-study}}

  \bibliography{book.bib,packages.bib}

\end{document}
